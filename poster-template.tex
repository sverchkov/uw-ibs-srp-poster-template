\documentclass[final,handout]{beamer}

% 36" x 48"
%\usepackage[size=custom,width=121.92,height=91.44,scale=1.5,debug]{beamerposter} % Landscape
%\usepackage[size=custom,width=91.44,height=121.92,scale=1.5,debug]{beamerposter} % Portrait

% 42" x 30"
%\usepackage[size=custom,width=106.68,height=76.2,scale=1.3,debug]{beamerposter} % Landscape

% 24" x 36"
\usepackage[size=custom,width=91.44,height=60.96,scale=1.5,debug]{beamerposter} % Landscape
%\usepackage[size=custom,width=60.96,height=91.44,scale=1.5,debug]{beamerposter} % Portrait

%% BIB
%\usepackage[style=authoryear,backend=biber,doi=false,isbn=false,url=false]{biblatex}
%\addbibresource{../library.bib}

%% FOR FIGURES
%\usepackage{epstopdf}

%% GRAPHICS
\usepackage{graphicx}
\graphicspath{{figures/}}
\DeclareGraphicsExtensions{.pdf,.png}

%% FOR TABLES
%\usepackage{colortbl}
%\usepackage{multirow}
%\usepackage{rotating}

%% SYMBOLS
%\DeclareMathOperator*{\argmax}{argmax}

%% COLORS
% It's good practice to only use a few colors. Defining them here makes it easy to change the color scheme by only changing the colors in one place. SoftRed is matched to the red in the UW logo, and StrongRed is a darker shade of the same hue.
\definecolor{StrongBlue}{HTML}{043C6B}
\definecolor{SoftBlue}{HTML}{3F8FD2}
\definecolor{StrongGreen}{HTML}{00733E}
\definecolor{SoftGreen}{HTML}{36D88E}
\definecolor{StrongRed}{HTML}{7B0F2A}
\definecolor{SoftRed}{HTML}{CC1945}

\setbeamercolor{structure}{fg=SoftRed}
\setbeamercolor{alerted text}{use=structure,fg=structure.fg}
\setbeamercolor{block title}{use=structure,fg=structure.bg,bg=structure.fg}
\setbeamercolor{block title example}{use=structure,fg=structure.fg,bg=structure.bg}

%% POSTER TEMPLATE DEFINITION
%%%%%%%%%%%%%%%%%%%%%%%%%%%%%%%%%%%%%%%%%%%%%%%%%%%%%%%%%%%%%%%%%%%%%%%%%%%%%%%%%%%%%%%%%%%%%%%%%%%%
\setbeamertemplate{navigation symbols}{}  % no navigation on a poster
\setbeamertemplate{headline}{  
  \leavevmode

  \begin{beamercolorbox}[wd=\paperwidth]{headline}
    \begin{columns}[totalwidth=\paperwidth]
      \column{.15\paperwidth}
       \begin{center}
         \includegraphics[trim={2.5cm 2cm 2.5cm 1cm},clip,height=28ex]{UWlogo_ctr_4c}
       \end{center}
      \column{.7\paperwidth}
        \begin{center}
        \usebeamercolor{title in headline}{\color{fg}\textbf{\LARGE{\inserttitle}}\\[1ex]}
        \usebeamercolor{author in headline}{\color{fg}\large{\insertauthor}\\[1ex]}
        \usebeamercolor{author in head/foot}{\color{fg}\small{\texttt{\insertsubtitle}}\\[1ex]}
        \usebeamercolor{institute in headline}{\color{fg}\large{\insertinstitute}\\[1ex]}     
        \end{center}
      \column{.15\paperwidth}
        \begin{center}
          \includegraphics[height=28ex]{WISCIENCE}
        \end{center}
    \end{columns}
  \end{beamercolorbox}

  \begin{beamercolorbox}[wd=\paperwidth]{lower separation line head}
    \rule{0pt}{3pt}
  \end{beamercolorbox}
}
%%%%%%%%%%%%%%%%%%%%%%%%%%%%%%%%%%%%%%%%%%%%%%%%%%%%%%%%%%%%%%%%%%%%%%%%%%%%%%%%%%%%%%%%%%%%%%%%%%%%

\title{Title goes here}
\subtitle{your email goes here} % Yes, it's a bit hacky but it's simple.
\author{Author List}
\institute{Program and/or department, University of Wisconsin--Madison}
\begin{document}
\begin{frame}{} %Everything is going to be inside this one frame
% Creates a columns environment (so that you can declare columns inside)
\begin{columns}[T, totalwidth=0.986\paperwidth]

% Creates a column of the specified width
\column{0.2\paperwidth}
 \vbox to .95\textheight{% Hack to make vfill work

% This changes the color of the block titles that follow, and some elements (like bullet points).
\setbeamercolor{structure}{fg=StrongBlue}

% Put every section of your poster inside a block. (Well, you don't have to, but it makes things pretty)
\begin{block}{Abstract \strut} % The \strut makes it so that the block titles are the same height, otherwise they'd only be as tall as they need to be to fit the text.
Some text
\begin{itemize}
 \item Bullet
 \item Points
 \item Here
\end{itemize}
\end{block}

\vfill % This creates vertical space between blocks, if needed.

%Let's make the next block red
\setbeamercolor{structure}{fg=SoftRed}

\begin{block}{Background and Significance \strut}
Some text
\begin{itemize}
 \item Bullet
 \item Points
 \item Here
\end{itemize}
\end{block}

\vfill

\begin{block}{Hypotheis or Research Question \strut}
 I sure hope this research is useful and impressive.
\end{block}

}% Make sure to close the vbox before the next column
% Makes the next column. You want them to sum up to less than 1 \paperwidth, for some breathing room.
\column{0.5\paperwidth}
\vbox to .95\textheight{% Hack to make vfill work

\begin{block}{Project / Experimental Design \strut}

Some text
\begin{itemize}
 \item Bullet
 \item Points
 \item Here
\end{itemize}
\end{block}

\vfill

\begin{block}{Results \strut}
\end{block}

}% Make sure to close the vbox before the next column
\column{0.2\paperwidth}
\vbox to .95\textheight{% Hack to make vfill work

\begin{block}{References \strut}
These often don't take this much space. You can maybe fit something more interesting in the third column.
\end{block}

\vfill

\begin{block}{Acknowledgements \strut}
\end{block}

}% Make sure to close the vbox
\end{columns} %Close every environment you open
\end{frame}
\end{document}